\section{Metodología}\label{sec:metodologia}

La experiencia comienza con la elección del ambiente operación en el cual se trabaja, a su vez presenta  los parámetros de 2680 MHz, un largo de 24 m y una longitud de onda correspondiente a 1,17 x 10$^{17}$m

Luego de esto se conecta la antena omnidireccional al generador de señales RF, configurando tal herramienta con la frecuencia anterior mencionada y una potencia de salida de 0 dBm. Posteriormente, se configura el analizador de espectros para la misma frecuencia central junto con un ancho de banda de 5 MHz.

Ya configuradas las herramientas a utilizar, se comienza la medición de los valores de potencia de la señal en el analizador de espectros utilizando como apoyo los marcadores. En primera instancia se realiza la medición cada 0.6 cm en un rango de 1 m a 3 m, luego se realiza la medición cada 1 m en el rango de 4 m a 20 m hasta llegar a las 50 mediciones.


Finalmente, cada dato obtenido mediante las herramientas usadas se van adjuntando en una hoja de cálculos para poder llevar a cabo un gráfico que represente de mejor medida los resultados.