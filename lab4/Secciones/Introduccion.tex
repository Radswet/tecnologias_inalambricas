\section{Introducción}\label{sec:introducción}
En el presente reporte, se detalla las actividades realizadas en la tercera experiencia de laboratorio. En primer lugar, se imparte la aplicación del método para caracterizar los efectos de propagación a gran escala en sistemas inalámbricos de comunicaciones para una banda de 2.69 GHz. Por otra parte, se caracteriza un ambiente inalámbrico en la banda de frecuencia de 2.69 GHz usando datos reales y el modelo ensombrecimiento log-shadowing.

Antes que todo, se debe definir algunos conceptos que se utilizan a lo largo de la experiencia.

\begin{itemize}
    \item \textbf{Multitrayectoria}:Fenómeno que consiste en la propagación de una onda por varios caminos diferentes, ocasionándose debido a los fenómenos de reflexión y difracción\cite{multi}.
   \item \textbf{Fenómeno Reflexión}:Cambio de dirección de una onda debido al choque con una superficie lisa y pulimentada sin cambiar el medio de propagación \cite{efect}.
   \item \textbf{Fenómeno Difracción}: Desviación de ondas alrededor de las esquinas de un obstáculo o a través de la abertura en la región de sombra del obstáculo\cite{efect}.
    \item \textbf{Modelos de propagación gran escala}:Modelos que predicen la potencia de la señal para cualquier distancia de separación entre el transmisor y el receptor.
      \item \textbf{Ensombrecimiento}:Fenómeno que ocurre debido al bloque de obstáculos en el camino de la señal, así como cambios en los objetos reflectores y dispersivos en el medio de transmisión \cite{shadow}.
      \item\textbf{Modelo Log - Shadowing}:Extensión del modelo de Friss que permite predecir la perdida de propagación para una amplia gama de entornos, a diferencia del modelo de Friss este no está restringido a una ruta clara sin obstrucciones entre el transmisor y el receptor\cite{tabla}.
    
\end{itemize}



Para poder representar las actividades descritas anteriormente, se presentan diversas secciones en el reporte donde se muestran tanto los resultados, análisis o conclusiones con respecto a la experiencia.